\chapter{Traitement des �quations}

\section{Introduction du chapitre}


\section{Les �quations non num�rot�es}

Exemple 1

$
a = b + c
$
\vspace{1cm}

Exemple 2

$$
a = b + c
$$
\vspace{1cm}


Exemple 3

\[
a = b + c
\]


\vspace{1cm}

Exemple 4

\begin{equation*}
\centering
a = b + c
\end{equation*}




\section{Les �quations num�rot�es}

\begin{equation}
a = b + c
\end{equation}


\section{les �quations sur plusieurs lignes}

Exemple 1

\begin{multline}
a + b + c + d + e + f
+ g + h + i
= j + k + l + m + n
\end{multline}

\vspace{1cm}

Exemple 2

\begin{equation}
a = b + c + d + e + f
+ g + h + i + j
+ k + l + m + n + o + p
\label{eq:equation_too_long}
\end{equation}

\section{�quations compliqu�es}

Exemple 1

\begin{equation}
a = \sum_{k=1}^n\sum_{\ell=1}^n
\sin \bigl(2\pi \, b_k \,
c_{\ell} \, d_k \, e_{\ell} \,
f_k \, g_{\ell} \, h \bigr)
\end{equation}


\vspace{1cm}
%Exemple 2


%\begin{numcases}{}
%\dot{x} = f(x,u)

%x+\dot{x} = h(x)
%\end{numcases}

%\vspace{1cm}
%Exemple 3


%\begin{equation*}
%\int_a^b f(x) \dd x = \int_a^b
%\ln\left(\frac{x}{2}\right)
%\dd x
%\end{equation*}



\section{Conclusion du chapitre}